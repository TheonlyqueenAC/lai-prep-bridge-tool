\documentclass[11pt]{article}
\usepackage[margin=0.75in]{geometry}
\usepackage{helvet}
\renewcommand{\familydefault}{\sfdefault}
\usepackage{xcolor}
\usepackage{titlesec}
\usepackage{enumitem}
\usepackage{hyperref}
\usepackage{tcolorbox}
\usepackage{amssymb}
\usepackage{array}
\usepackage{multirow}

\titleformat{\section}{\Large\bfseries\color{blue!70!black}}{\thesection}{1em}{}
\titleformat{\subsection}{\large\bfseries\color{blue!50!black}}{\thesubsection}{1em}{}
\titleformat{\subsubsection}{\normalsize\bfseries\color{blue!40!black}}{\thesubsubsection}{1em}{}

\tcbuselibrary{breakable}

\begin{document}

\begin{center}
{\Huge\bfseries Supplementary File S5}\\[0.3cm]
{\LARGE Clinical Decision Flowchart}\\[0.5cm]
{\large Step-by-Step Visual Guide for LAI-PrEP Bridge Period Navigation}\\[0.3cm]
{\normalsize Adrian C. Demidont and Kandis Backus}\\[0.2cm]
{\small\textit{Viruses} Journal Supplementary Materials}
\end{center}

\vspace{0.5cm}

\section*{Purpose of This Flowchart}

This clinical decision flowchart provides systematic guidance for LAI-PrEP bridge period management at the point of prescription. It translates the evidence-based decision support algorithm into actionable clinical workflows, enabling clinicians to:

\begin{itemize}
\item Rapidly identify patients for same-day switching protocols
\item Assess population-specific risks and barriers
\item Stratify patients by predicted success rate
\item Select appropriate evidence-based interventions
\item Implement structured follow-up protocols
\end{itemize}

\textbf{Critical Insight}: Without systematic intervention, 47\% of LAI-PrEP prescriptions do not result in injection initiation. This flowchart addresses that implementation gap.

\section{Step 1: Oral PrEP Status Assessment}

\begin{tcolorbox}[colback=blue!5!white,colframe=blue!75!black,title=\textbf{INITIAL TRIAGE QUESTION},breakable]
\textbf{Is the patient currently taking oral PrEP?}

This is the single most important clinical decision point that determines bridge period pathway.
\end{tcolorbox}

\subsection{YES: Patient on Oral PrEP → Expedited Pathway}

\begin{tcolorbox}[colback=green!10!white,colframe=green!75!black,title=\textbf{Secondary Question: Recent HIV Test?},breakable]

\subsubsection{If YES (HIV test within 7 days):}

\begin{tcolorbox}[colback=green!20!white,colframe=green!75!black,title=* PRIORITY 1: Same-Day Switching Protocol]
\textbf{Predicted Success: 90\%} \\
\textbf{Bridge Period: 0--3 days}

\textbf{Actions at Prescription Visit:}
\begin{itemize}[leftmargin=*]
\item $\checkmark$ Inject TODAY (preferred) or within 3 days maximum
\item $\checkmark$ Submit insurance authorization same day
\item $\checkmark$ Document switch in medical record
\item $\checkmark$ Schedule next injection (2 or 6 months)
\item $\checkmark$ Provide injection site care instructions
\end{itemize}

\textbf{Evidence}: OPERA cohort (n=302), Trio Health (n=146) demonstrated 85--90\% success with same-day protocols.

\textbf{Key Point}: Do NOT make these patients wait. They are already engaged and adherent.
\end{tcolorbox}

\subsubsection{If NO (HIV test >7 days ago or never):}

\begin{tcolorbox}[colback=green!15!white,colframe=green!75!black,title=* PRIORITY 2: Rapid Transition Protocol]
\textbf{Predicted Success: 85--90\%} \\
\textbf{Bridge Period: 7--14 days}

\textbf{Actions at Prescription Visit:}
\begin{itemize}[leftmargin=*]
\item $\checkmark$ Order STAT HIV testing (same day if possible)
\item $\checkmark$ Submit insurance authorization TODAY (do not wait for test)
\item $\checkmark$ Schedule injection for next week (tentative)
\item $\checkmark$ Confirm injection once negative test result received
\item $\checkmark$ Continue oral PrEP until injection
\item $\checkmark$ Set text reminders for test and injection appointments
\end{itemize}

\textbf{Goal}: Minimize wait time to preserve oral PrEP adherence momentum.
\end{tcolorbox}
\end{tcolorbox}

\subsection{NO: Patient NOT on Oral PrEP → Standard Pathway}

\begin{tcolorbox}[colback=orange!10!white,colframe=orange!75!black]
Proceed to \textbf{STEP 2: Population \& Barrier Assessment}

This pathway requires systematic risk assessment and intervention planning.
\end{tcolorbox}

\section{Step 2: Population \& Barrier Assessment}

\subsection{Population Identification}

Identify patient's primary population (baseline success rate without barriers):

\begin{table}[h]
\centering
\begin{tabular}{|l|c|l|}
\hline
\textbf{Population} & \textbf{Baseline Success} & \textbf{Evidence Source} \\
\hline
MSM & 55\% & HPTN 083 \\
Cisgender Women & 45\% & HPTN 084, PURPOSE-1 \\
Transgender Women & 50\% & HPTN 083, PURPOSE-2 \\
Adolescents (16--24) & 35\% & PURPOSE-1, oral PrEP \\
PWID & 25\% & Oral PrEP cascade \\
Pregnant/Lactating & 45\% & PURPOSE-1 \\
General Population & 53\% & Real-world cohorts \\
\hline
\end{tabular}
\end{table}

\subsection{Barrier Assessment Checklist}

Check ALL barriers that apply. Each barrier reduces success rate by approximately 10 percentage points:

\begin{tcolorbox}[colback=yellow!10!white,colframe=yellow!75!black,breakable]
\textbf{Structural Barriers:}

\begin{itemize}[label=$\square$,leftmargin=*]
\item Transportation barriers (no reliable access to clinic)
\item Childcare needs (cannot attend appointments without childcare)
\item Housing instability (homeless or unstable housing)
\item Insurance delays expected (prior authorization typically >2 weeks)
\item Scheduling conflicts (work/school during clinic hours)
\end{itemize}

\textbf{Interpersonal \& Systemic Barriers:}

\begin{itemize}[label=$\square$,leftmargin=*]
\item Medical mistrust (history of negative healthcare experiences)
\item Privacy concerns (disclosure fears, confidentiality needs)
\item Healthcare discrimination (experienced/anticipated discrimination)
\item Competing priorities (other urgent health/life needs)
\item Limited healthcare navigation experience (new to system)
\end{itemize}

\textbf{Population-Specific Barriers:}

\begin{itemize}[label=$\square$,leftmargin=*]
\item Legal/criminalization concerns (PWID, sex work, immigration)
\item Lack of government identification
\item Active substance use (interfering with appointment attendance)
\end{itemize}
\end{tcolorbox}

\subsection{Calculate Adjusted Success Rate}

\begin{tcolorbox}[colback=blue!10!white,colframe=blue!75!black,title=\textbf{Calculation Formula}]
\textbf{Adjusted Success Rate = Baseline Rate -- (10\% × Number of Barriers)}

\textbf{Examples:}
\begin{itemize}
\item MSM (55\%) with 1 barrier (transportation) = 45\% success
\item Cisgender woman (45\%) with 3 barriers (transport, childcare, mistrust) = 15\% success
\item PWID (25\%) with 4 barriers = Cannot go below 0\% (use 5--10\% estimate)
\end{itemize}

\textbf{Note}: This is a simplified clinical calculation. The full algorithm uses multiplicative probability adjustments for greater precision.
\end{tcolorbox}

\section{Step 3: Risk Categorization \& Intervention Selection}

Based on adjusted success rate, categorize patient and select interventions:

\subsection{Low Risk: Adjusted Success >70\%}

\begin{tcolorbox}[colback=green!10!white,colframe=green!75!black,breakable]
\textbf{Risk Level}: Low \\
\textbf{Predicted Success}: 70--85\%

\textbf{Standard Protocols:}
\begin{itemize}[leftmargin=*]
\item Text/email reminders for appointments
\item Expedited HIV testing (within 3--5 days)
\item Standard insurance authorization process
\item Patient education materials
\end{itemize}

\textbf{Follow-up}: Brief check-in call 1 week post-prescription
\end{tcolorbox}

\subsection{Moderate Risk: Adjusted Success 50--69\%}

\begin{tcolorbox}[colback=yellow!15!white,colframe=yellow!75!black,breakable]
\textbf{Risk Level}: Moderate \\
\textbf{Predicted Success}: 60--75\% (with interventions)

\textbf{Enhanced Protocols:}
\begin{itemize}[leftmargin=*]
\item \textbf{Assign patient navigator} (2--3 contacts during bridge period)
\item Text/email/phone reminders
\item Expedited/same-day HIV testing
\item Address 1--2 key barriers with targeted interventions
\item Insurance support (tracking, appeals if needed)
\end{itemize}

\textbf{Barrier-Specific Interventions:}
\begin{itemize}[leftmargin=*]
\item Transportation → Uber/Lyft vouchers, mileage reimbursement
\item Scheduling → Extended hours, weekend appointments
\item Insurance → Pre-authorization assistance, patient assistance programs
\end{itemize}

\textbf{Follow-up}: Navigator contact within 24--48 hours, then weekly
\end{tcolorbox}

\subsection{High Risk: Adjusted Success 30--49\%}

\begin{tcolorbox}[colback=orange!15!white,colframe=orange!75!black,breakable]
\textbf{Risk Level}: High \\
\textbf{Predicted Success}: 40--60\% (with intensive interventions)

\textbf{Intensive Interventions Required:}
\begin{itemize}[leftmargin=*]
\item \textbf{Navigator assignment} (MANDATORY -- minimum 3 contacts)
\item Accelerated HIV testing (same-day rapid test if possible)
\item Transportation support (vouchers, rides, mileage)
\item Barrier-specific intensive support:
\begin{itemize}
\item Childcare vouchers or on-site childcare
\item Flexible scheduling (early/late/weekend)
\item Insurance advocacy and appeals
\item Peer navigation (if available)
\end{itemize}
\item Close follow-up (every 2--3 days)
\item Case conference if barriers persist
\end{itemize}

\textbf{Multiple Intervention Modalities}: Combine structural support (transportation, childcare) with interpersonal support (navigation, peer support)
\end{tcolorbox}

\subsection{Very High Risk: Adjusted Success <30\%}

\begin{tcolorbox}[colback=red!15!white,colframe=red!75!black,breakable]
\textbf{Risk Level}: Very High \\
\textbf{Predicted Success}: 20--40\% (with maximum interventions)

\textbf{Maximum Intensity Interventions:}
\begin{itemize}[leftmargin=*]
\item \textbf{Intensive navigation} (multiple contacts weekly)
\item \textbf{Peer navigator} (if available, especially for PWID, transgender populations)
\item \textbf{Harm reduction integration} (for PWID)
\item \textbf{Mobile/outreach services} (bring services to patient)
\item \textbf{Low-barrier protocols}:
\begin{itemize}
\item No ID requirement
\item Flexible scheduling
\item Home visits if needed
\item Telehealth options
\end{itemize}
\item Address ALL identified barriers simultaneously
\item Daily contact first week, then every 2--3 days
\item Case management beyond bridge period
\end{itemize}

\textbf{Critical}: These patients require healthcare system-level support, not just individual interventions. Consider alternative care delivery models.
\end{tcolorbox}

\section{Step 4: Evidence-Based Intervention Library}

Select interventions based on specific barriers and population:

\begin{table}[h]
\small
\centering
\begin{tabular}{|p{4cm}|p{5cm}|c|}
\hline
\textbf{Intervention} & \textbf{Target Barriers/Populations} & \textbf{Effect} \\
\hline
\multicolumn{3}{|c|}{\textbf{High-Impact Interventions (>15\% improvement)}} \\
\hline
Same-day switching & Oral PrEP patients & +25\% \\
Patient navigation & All barriers, all populations & +15\% \\
Peer navigation & PWID, transgender, MSM & +18\% \\
Harm reduction integration & PWID, substance use & +18\% \\
\hline
\multicolumn{3}{|c|}{\textbf{Moderate-Impact Interventions (10--15\%)}} \\
\hline
Transportation support & Transportation barriers & +12\% \\
Accelerated testing & HIV testing delays & +12\% \\
Anti-discrimination protocols & Discrimination experiences & +12\% \\
Low-barrier protocols & Multiple barriers, PWID & +12\% \\
Childcare support & Childcare needs & +10\% \\
Insurance support & Insurance delays & +10\% \\
Prenatal integration & Pregnant/lactating & +10\% \\
Medical mistrust intervention & Medical mistrust & +10\% \\
\hline
\multicolumn{3}{|c|}{\textbf{Supportive Interventions (5--10\%)}} \\
\hline
Flexible scheduling & Scheduling conflicts & +6\% \\
Text/email reminders & All patients & +8\% \\
Confidentiality protections & Privacy concerns, adolescents & +8\% \\
Pregnancy counseling & Pregnant/lactating & +8\% \\
Mobile delivery & Housing instability, PWID & +8\% \\
Cultural competency & Discrimination, mistrust & +7\% \\
Telehealth options & Transportation, rural & +5\% \\
Community partnerships & All populations & +5\% \\
Extended clinic hours & Scheduling conflicts & +5\% \\
Same-day appointments & Competing priorities & +5\% \\
\hline
\end{tabular}
\end{table}

\section{Step 5: Special Population Protocols}

\subsection{PWID Fast Track}

\begin{tcolorbox}[colback=red!10!white,colframe=red!75!black,title=\textbf{People Who Inject Drugs: Alternative Care Model Required},breakable]
\textbf{Critical Insight}: Traditional clinic-based care results in <10\% success for PWID. An alternative approach is essential.

\textbf{Required Elements:}
\begin{itemize}[leftmargin=*]
\item \textbf{MUST} partner with syringe services program (SSP) or harm reduction program
\item Bring ALL services to the patient (co-locate at SSP site)
\item Use peer navigators with lived experience
\item Low-barrier protocols:
\begin{itemize}
\item No government ID required
\item No abstinence requirements
\item Flexible appointment times
\item No-show tolerant (immediate rescheduling)
\end{itemize}
\item Rapid HIV testing at SSP site (same-day results)
\item Mobile delivery if SSP partnership unavailable
\item Integrate with medication-assisted treatment (MAT)
\item Address housing and food insecurity simultaneously
\end{itemize}

\textbf{Expected Outcome}: 30--40\% success (compared to <10\% in traditional clinic)

\textbf{Evidence}: Harm reduction PrEP literature, oral PrEP PWID cascade studies
\end{tcolorbox}

\subsection{Adolescent Fast Track}

\begin{tcolorbox}[colback=orange!10!white,colframe=orange!75!black,title=\textbf{Adolescents (16--24): Youth-Specific Approach},breakable]
\textbf{Key Barriers}: Transportation dependence, privacy concerns, limited healthcare navigation experience

\textbf{Required Elements:}
\begin{itemize}[leftmargin=*]
\item Youth-specific navigator (ESSENTIAL -- trained in adolescent development)
\item Transportation without parental involvement (vouchers, youth-friendly transit)
\item Confidential scheduling and communication
\item School-friendly appointment times (after school, early morning, weekends)
\item Bundle appointments (test + inject same day when possible)
\item Text-based communication (preferred by adolescents)
\item Privacy protections (manage insurance EOBs, parental notifications)
\item Brief, focused visits (adolescent attention span)
\end{itemize}

\textbf{Expected Outcome}: 35--50\% success with navigation (vs. <20\% without)

\textbf{Evidence}: PURPOSE-1 adolescent cohort, oral PrEP adolescent cascade
\end{tcolorbox}

\subsection{Oral PrEP Patients: Your Easiest Win}

\begin{tcolorbox}[colback=green!10!white,colframe=green!75!black,title=\textbf{Current Oral PrEP Users: Highest Success Opportunity},breakable]
\textbf{Critical Message}: These are your highest-success patients. Do NOT let them fall through cracks.

\textbf{Streamlined Protocol:}
\begin{itemize}[leftmargin=*]
\item Recent HIV test (within 7 days)? → INJECT TODAY
\item No recent test? → Order test, inject within 7 days maximum
\item Same-day insurance authorization (do not delay)
\item Minimal wait time (preserve adherence momentum)
\item Build on existing provider relationship
\item Continue oral PrEP until injection (if needed)
\end{itemize}

\textbf{Expected Outcome}: 85--90\% success

\textbf{Key Point}: Every day of delay increases risk of oral PrEP discontinuation and loss to follow-up.
\end{tcolorbox}

\section{Step 6: Implementation \& Follow-Up Timeline}

\subsection{Day 0: Prescription Visit}

\begin{tcolorbox}[colback=blue!5!white,colframe=blue!75!black,breakable]
\textbf{Essential Actions at Prescription:}

\begin{itemize}[label=$\square$,leftmargin=*]
\item Complete barrier assessment (use checklist in Step 2)
\item Calculate adjusted success rate and risk category
\item Select interventions based on this flowchart
\item Order HIV testing (expedited/STAT)
\item Submit insurance authorization SAME DAY (critical!)
\item Assign patient navigator if moderate to very high risk
\item Provide transportation voucher if barrier identified
\item Schedule tentative injection appointment
\item Set up text/email reminders
\item Give patient clear timeline and expectations
\item Provide patient handout (Supplementary File S2)
\item Document barriers and intervention plan in medical record
\end{itemize}

\textbf{Time Investment}: 15--20 minutes for comprehensive assessment
\end{tcolorbox}

\subsection{Day 1: Next Business Day}

\begin{tcolorbox}[colback=blue!5!white,colframe=blue!75!black]
\textbf{Navigator Actions:}

\begin{itemize}[label=$\square$,leftmargin=*]
\item Contact patient (phone or text)
\item Confirm understanding and motivation
\item Address any new barriers that have emerged
\item Confirm all appointment times
\item Check insurance authorization status
\item Problem-solve any concerns
\end{itemize}
\end{tcolorbox}

\subsection{Days 2--7: Testing Phase}

\begin{tcolorbox}[colback=blue!5!white,colframe=blue!75!black,breakable]
\textbf{Critical Period:}

\begin{itemize}[label=$\square$,leftmargin=*]
\item HIV testing completed (ideally within 3--5 days)
\item Results reviewed same day or next business day
\item Navigator provides results and confirms injection appointment
\item Text reminders sent (48 hours and 24 hours before injection)
\item Insurance authorization confirmed or escalated if denied
\item Address any barriers to injection appointment
\end{itemize}

\textbf{High-Risk Patients}: Contact every 2--3 days during this period
\end{tcolorbox}

\subsection{Days 7--28: Injection Window}

\begin{tcolorbox}[colback=green!10!white,colframe=green!75!black,title=\textbf{TARGET: FIRST INJECTION},breakable]
\textbf{Injection Visit:}

\begin{itemize}[label=$\square$,leftmargin=*]
\item Administer first injection
\item Patient education on injection-site reactions (common, self-limited)
\item Provide contact information for questions/concerns
\item Schedule next injection appointment (2 months for cabotegravir, 6 months for lenacapavir)
\item Hand off to retention/persistence program
\item Document outcome in tracking system
\item Celebrate success with patient!
\end{itemize}

\textbf{Goal Timeline:}
\begin{itemize}
\item Oral PrEP patients: 0--14 days
\item New patients, low risk: 14--21 days
\item New patients, moderate/high risk: 21--28 days
\end{itemize}
\end{tcolorbox}

\subsection{If Patient Misses Appointment}

\begin{tcolorbox}[colback=red!10!white,colframe=red!75!black,title=\textbf{SAME-DAY RESPONSE REQUIRED},breakable]
\textbf{Immediate Actions (Day of Miss):}
\begin{itemize}[leftmargin=*]
\item Call patient immediately
\item Identify barrier that caused missed appointment
\item Problem-solve barrier with patient
\item Reschedule for ASAP (within 3 days if possible)
\item Offer additional support (transportation, flexible timing, etc.)
\end{itemize}

\textbf{If Cannot Reach:}
\begin{itemize}[leftmargin=*]
\item Send text message
\item Try alternate contact method (email, secondary phone)
\item Attempt contact daily for 3 days minimum
\item Consider home visit or outreach for very high-risk patients
\item Do NOT give up after one attempt
\end{itemize}

\textbf{Key Message}: Missing one appointment is NOT failure. Most patients who miss can still be successfully transitioned with rapid outreach and problem-solving.
\end{tcolorbox}

\section{Clinical Pearls}

\begin{tcolorbox}[colback=yellow!10!white,colframe=yellow!75!black,title=\textbf{Top 10 Implementation Insights},breakable]

\textbf{1. The \#1 Thing}: Identify oral PrEP patients and transition them FAST. This is your easiest win and highest success rate.

\textbf{2. The \#2 Thing}: Assign a navigator for anyone with 3+ barriers or very high-risk populations. Navigation is the single most effective intervention.

\textbf{3. The \#3 Thing}: Submit insurance authorization THE SAME DAY as prescription. Do not wait for HIV test results. Delays here cause 10--15\% attrition.

\textbf{4. What NOT to Do}: Prescribe and hope. Without proactive intervention, 47\% will not initiate. Passive approaches fail.

\textbf{5. PWID Specific}: Traditional clinic-based care will fail for PWID. You MUST use harm reduction approach with SSP integration. There is no successful traditional alternative.

\textbf{6. Timeline Matters}: Every extra day increases attrition risk. Aim for <14 days for oral PrEP transitions, <28 days for new patients.

\textbf{7. Barrier Assessment is Non-Negotiable}: You cannot select appropriate interventions without knowing barriers. Budget 5 minutes for systematic assessment.

\textbf{8. Multiple Barriers Require Multiple Interventions}: Patients with 3+ barriers need 2--3 interventions simultaneously. Single interventions are insufficient.

\textbf{9. Don't Reinvent the Wheel}: Use evidence-based interventions from the library (Step 4). Effectiveness is proven; customize implementation to your setting.

\textbf{10. Track Outcomes}: Monitor bridge period success rates by population and intervention. Use data to improve your local protocols continuously.
\end{tcolorbox}

\section{Evidence Base}

This flowchart is based on:

\begin{itemize}
\item \textbf{Clinical Trials}: HPTN 083 (n=4,566 MSM/transgender women), HPTN 084 (n=3,224 cisgender women), PURPOSE trials (n=10,761 across multiple populations)
\item \textbf{Real-World Implementation}: CAN Community Health Network (n=302), OPERA cohort, Trio Health, SPAN clinics
\item \textbf{Barrier Literature}: PrEP cascade studies, implementation science, structural barrier research
\item \textbf{Intervention Evidence}: Systematic reviews and meta-analyses of navigation (k=23 RCTs), harm reduction integration, peer support
\item \textbf{Computational Validation}: Decision support algorithm validated at UNAIDS scale (21.2 million patients), 100\% unit test pass rate
\end{itemize}

\section*{Usage Instructions}

\textbf{For Clinicians}:
\begin{itemize}
\item Print this flowchart and keep in LAI-PrEP prescription area
\item Use at EVERY LAI-PrEP prescription to systematically assess and intervene
\item Document selected interventions in medical record
\item Share with nursing staff and navigators for care coordination
\end{itemize}

\textbf{For Clinic Administrators}:
\begin{itemize}
\item Train all prescribers on flowchart use
\item Ensure navigation resources available for moderate/high-risk patients
\item Track bridge period outcomes by population and risk level
\item Use data to refine local protocols and resource allocation
\end{itemize}

\textbf{For Researchers}:
\begin{itemize}
\item Test flowchart effectiveness in prospective implementation studies
\item Validate risk stratification accuracy in diverse settings
\item Evaluate cost-effectiveness of tiered intervention approach
\item Document adaptations needed for specific contexts
\end{itemize}

\vspace{1cm}

\begin{center}
\textit{Use this flowchart at every LAI-PrEP prescription to systematically identify risks and implement evidence-based interventions.}

\vspace{0.3cm}

\textit{Based on: Demidont, A.C.; Backus, K.V. Bridging the Gap: Computational Validation of Clinical Decision Support Algorithm for Long-Acting Injectable PrEP Bridge Period Navigation. Viruses 2025.}
\end{center}

\end{document}